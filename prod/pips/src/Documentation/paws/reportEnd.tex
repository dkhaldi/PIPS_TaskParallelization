
The major goal of the project was to create a lightweight WEB
application that allows to use PIPS without installing it. Other goals
were to provide different types of ways of using PIPS, which were
depending on the user's level of familiarity with PIPS. Two lists of
constraints and functional requirements can be found in
Section~\ref{design_contraints} and \ref{design_requirements}.

To achieve those goals, a WEB application was designed and
developed. It presents PIPS in two modes, described in
Section~\ref{paws_project}. The content of all pages of those modes is
generated dynamically and is based on the structure of the directories
and on the content of description files. Thanks to this design, the
reconfiguration of PaWS is very easy and does not require knowledge of
the Pylons framework.

The implementation contains also a lot of user-friendly features (see
requirements in Section~\ref{design_requirements} and components used
\ref{components_used}), like creating, displaying and zooming graphs,
resizing font size (according to user preferences), uploading user
files (also archive .zip files), saving results of the operations on
user's machine or printing them.

The main goal for the future is to implement a third mode, full
control, which will enable user to create graphically his/her own PIPS
script. PaWS should also have more tests - currently there are no
tests for its graphical interface. Another new utility would be a
Pyps/Tpips code generation with all the properties, analyses and
phases used for advanced mode. Also making it possible to save and
print the dependence graphs might be useful.

Other small, possible improvements are:
\begin{itemize}
  \item Adding the scale for a slider in the demonstation module.
  \item Controlling the response time from the server.
  \item Integrating the IR Navigator \cite{irnavigator}.
\end{itemize}

The current version of the PaWS framework meets all the design
constraints and requirements imposed at the beginning. It can also be
easily extended with the new functionality.
