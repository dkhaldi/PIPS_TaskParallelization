%%
%% $Id: text.tex 23065 2016-03-02 09:05:50Z coelho $
%%
%% Copyright 1989-2016 MINES ParisTech
%%
%% This file is part of PIPS.
%%
%% PIPS is free software: you can redistribute it and/or modify it
%% under the terms of the GNU General Public License as published by
%% the Free Software Foundation, either version 3 of the License, or
%% any later version.
%%
%% PIPS is distributed in the hope that it will be useful, but WITHOUT ANY
%% WARRANTY; without even the implied warranty of MERCHANTABILITY or
%% FITNESS FOR A PARTICULAR PURPOSE.
%%
%% See the GNU General Public License for more details.
%%
%% You should have received a copy of the GNU General Public License
%% along with PIPS.  If not, see <http://www.gnu.org/licenses/>.
%%
\documentclass{article}

\usepackage[latin1]{inputenc}
\usepackage{newgen_domain}
\usepackage[backref,pagebackref]{hyperref}

\begin{document}
\sloppy

The domain text is used to factor out of PIPS prettyprinter all layout
considerations. It does not depend on PIPS internal representation.

The idea is to have higher level structures to manipulate text source more
efficiently. Since a text is mainly a list of sentences and sentences a
list of words, movings sentences or concatenating sentence is more
efficient than working directly on arrays of char, since we just work on
the sentence list.

The final layout (with indentation...) can be delayed through the use of
\verb|unformatted| sentences.


\domain{Text = sentences:sentence*}
{}

A text is a list of sentences.

\domain{Sentence = formatted:string + unformatted}
{}

Each sentence can be either formatted and the corresponding string
must be printed out exactly as it is, or unformatted and line breaks
and indentation must be generated.

\domain{Unformatted = label:string x number:int x extra\_margin:int x  words:string*}
{}

Because of the Fortran origin, but it is also useful in C, a label can
be defined as a string for each sentence. The number field is the
statement number. The indentation is defined by field extra\_margin,
extra because of Fortran~77 fixed input fields. Finally a list of
words define the content of the sentence. The linefeed can be inserted
between two words and the continuation handled if necessary.

Since in C we also have some \verb|#pragma| between the label and the
instruction, there is a little semantics gap compared to Fortran here...

FI: I assume that the text library depends on the output language.

FI: Fabien Coelho has developped another interface to output source code.

\end{document}
